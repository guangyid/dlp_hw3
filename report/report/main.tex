% Document from Zhihu https://zhuanlan.zhihu.com/p/496462136
% Refer to the author in your mind
\documentclass[10pt,a4paper]{article}% 文档格式
\usepackage{ctex,hyperref}% 输出汉字
\usepackage{times}% 英文使用Times New Roman
\title{\fontsize{18pt}{27pt}\selectfont% 小四字号,1.5倍行距
	{\heiti% 黑体 
	DLP第三次实验报告\\{\rightline{——TinaFace\cite{DBLP:journals/corr/abs-2011-13183}复现及改进}}
	}}% 题目

\author{\fontsize{12pt}{18pt}\selectfont% 小四字号,1.5倍行距
	{\fangsong% 仿宋
	\rightline{韩广意PB20030835}
	}\\% 标题栏脚注main.tex
	\fontsize{10.5pt}{15.75pt}\selectfont% 五号字号,1.5倍行距
}
\date{}
% \date{\today}% 日期(这里避免生成日期)

\usepackage{amsmath,amsfonts,amssymb}% 为公式输入创造条件的宏包
\usepackage{graphicx}% 图片插入宏包
\usepackage{subfigure}% 并排子图
\usepackage{float}% 浮动环境,用于调整图片位置
\usepackage[export]{adjustbox}% 防止过宽的图片
\usepackage{bibentry}
\usepackage{natbib}% 以上2个为参考文献宏包
\usepackage{abstract}% 两栏文档,一栏摘要及关键字宏包
% \renewcommand{\abstracttextfont}%{\fangsong}% 摘要内容字体为仿宋
\renewcommand{\abstractname}{\textbf{摘\quad 要}}% 更改摘要二字的样式
\usepackage{xcolor}% 字体颜色宏包
\newcommand{\red}[1]{\textcolor[rgb]{1.00,0.00,0.00}{#1}}
\newcommand{\blue}[1]{\textcolor[rgb]{0.00,0.00,1.00}{#1}}
\newcommand{\green}[1]{\textcolor[rgb]{0.00,1.00,0.00}{#1}}
\newcommand{\darkblue}[1]
{\textcolor[rgb]{0.00,0.00,0.50}{#1}}
\newcommand{\darkgreen}[1]
{\textcolor[rgb]{0.00,0.37,0.00}{#1}}
\newcommand{\darkred}[1]{\textcolor[rgb]{0.60,0.00,0.00}{#1}}
\newcommand{\brown}[1]{\textcolor[rgb]{0.50,0.30,0.00}{#1}}
\newcommand{\purple}[1]{\textcolor[rgb]{0.50,0.00,0.50}{#1}}% 为使用方便而编辑的新指令
\usepackage{url}% 超链接
\usepackage{bm}% 加粗部分公式
\usepackage{multirow}
\usepackage{booktabs}
\usepackage{epstopdf}
\usepackage{epsfig}
\usepackage{longtable}% 长表格
\usepackage{supertabular}% 跨页表格
\usepackage{algorithm}
\usepackage{algorithmic}
\usepackage{changepage}% 换页
\usepackage{makecell}
\usepackage{enumerate}% 短编号
\usepackage{caption}% 设置标题
\usepackage{stfloats}
\captionsetup[figure]{name=\fontsize{10pt}{15pt}\selectfont Figure}% 设置图片编号头
\captionsetup[table]{name=\fontsize{10pt}{15pt}\selectfont Table}% 设置表格编号头
\usepackage{indentfirst}% 中文首行缩进
\usepackage[left=3.17cm,right=3.17cm,top=2.54cm,bottom=2.54cm]{geometry}% 页边距设置
\renewcommand{\baselinestretch}{1}% 定义行间距(1.5)
\usepackage{fancyhdr} %设置全文页眉、页脚的格式
\pagestyle{fancy}
\hypersetup{colorlinks=true,linkcolor=black}% 去除引用红框,改变颜色
 
\begin{document}% 以下为正文内容
	\maketitle% 产生标题,没有它无法显示标题
	\lhead{}% 页眉左边设为空
	\chead{}% 页眉中间设为空
	\rhead{}% 页眉右边设为空
	\lfoot{}% 页脚左边设为空
	\cfoot{\thepage}% 页脚中间显示页码
	\rfoot{}% 页脚右边设为空
	 
	\section{Tinaface复现}
	\subsection{网络结构}
	\begin{figure*}[h]% 插入一张图片,H表示浮动环境下的here
		\centering
		\begin{minipage}{1\textwidth}% 小页面尺寸,可自行调节
			\centering
			\includegraphics[width=1.0\textwidth]{architecture.png}% 图片名称(图片需与tex文件在同一文件夹)
			\caption{\fontsize{10pt}{15pt}\selectfont 
			TinaFace的模型架构。(a)特征提取器:ResNet-50和6级特征金字塔网络,P4、P5使用了DCN网络,提取输入图像的多尺度特征。(b)Inception模块增强接受野。(c)分类头:5层FCN用于锚框的分类。
			(d)回归头:5层FCN,用于锚框回归到地真框。(e) IoU感知头:用于IoU预测的单个卷积层。
		}% 图例
		\end{minipage}
	\end{figure*}	

	\subsection{实现细节}
	\subsubsection{原论文实验条件}
	作者在 mmdetection 库的基础上将工程结构进行了一定修改\footnote{代码仓库地址:https://github.com/Media-Smart/vedadet/tree/main/configs/trainval/tinaface},
	使其适应TensorRT。作者论文的实验部分提到了以下对复现结果有影响的配置,我根据所能使用的硬件条件进行了自己的选择。\\
	\subsubsection*{归一化}
	作者考虑到批量归一化(BN)随着批处理规模的减小性能会下降,当批处理规模小于4时,由于批统计估计不准确而导致模型性能下降。
	同时考虑到大显存的gpu并不广泛使用,提供了将所有BN层网络用Group Normalization(GN) 替代的方案。在复现过程中也使用GN方法。\\
	\subsubsection*{训练配置}
	作者在3个GeForce GTX 1080 Ti上使用批量为3*4的SGD优化器(动量0.9,重量衰减5e-4)对模型进行训练。
	学习速率的过程是在630个epoch中每30个epoch用余弦衰减规则从3.75e-3退火到3.75e-5。
	在前500次迭代中,学习率从3.75e-4线性升到3.75e-3。\\
	\subsubsection*{测试设置}
	作者在评估环节为了提高评分,最后的评估结果是使用了测试时增强(Test Time Augmentation, TTA)的,
	但是考虑到TTA只是提高评分的tricks,不能说明模型设计的优越性且花费更多的计算资源,
	我在复现评估时没有使用TTA方法,只将baseline的评估结果与作者对应的结果进行对比。\\
	\subsubsection*{环境配置}
	Linux;Python 3.7+;PyTorch 1.6.0 or higher;CUDA 10.2 or higher

	\subsection{复现过程}
	\subsubsection*{环境配置}
	最终在Bitahub上使用的环境为	Ubuntu;Python 3.6;PyTorch 1.6.0;CUDA 11.0

	因为需要对C++和Cuda进行编译,所以对torch和cudatoolkit的版本要求并不像作者所说的那么宽松。torch的版本不能过高,在高版本中的torch中移除了THC/THC.h文件(替换为ATen/ATen.h);
	cuda的版本不能低,不然也无法编译。因为不懂dockerfile,长时间无法构建出一个可以使用合适版本的nvcc编译的镜像,后来在助教的帮助下解决。\\
	\subsubsection*{数据集加载}
	在本地运行时数据集是以小文件存储读取的,但是上传到Bitahub时管理员告知需要对图片进行聚合,所以将数据集聚合成lmdb数据集,但是似乎自己写的lmdb数据集的读取过于简单粗暴了,
	在本地上训练一个epoch需要大概20分钟,但是到Bitahub变成30分钟了,观察GPU的Memory情况,发现波动较大,初步推测是数据加载成为瓶颈,但是也不排除其它硬件条件差异和版本差异影响。\\
	\subsubsection*{训练过程}
	在两块1080Ti上训练,每块batchsize = 4,其余设置未更改。因为数据集较大,训练一个epoch大致要30min,如果真的要训练满630个epoch消耗算力过多,为了不占用其他同学的资源,我训练了105
	个epoch,消耗算力70点。

	\subsection{复现结果分析}
	\begin{center}
	\begin{tabular}{|c| c||c|c|c|}
		\hline
		模型 & epochs & AP(easy)& AP(middle) & AP(hard)\\
		\hline
		作者(有TTA) & 630 & 0.970 & 0.963 &	0.934\\
		\hline
		作者(无TTA) & 630 & 0.963 & 0.957 &	0.930\\
		\hline
		hgy(无TTA) & 105 & 0.940 & 0.931 &	0.899\\
		\hline
	\end{tabular}
	\end{center}
	\subsubsection{loss曲线}
	见左下 Figure 2,可以发现在epoch = 105时,损失函数下降趋势变缓但是仍在逐渐下降,反映到训练效果上应该是:较容易学习的特征已经学习完成,但是较难学习的内容仍在学习中。推测 WIDER FACE 数据集
	中测试结果应该是easy、middle、hard部分距离作者最终成品的差距越大,验证结果确实如此,见上。
	\begin{figure}[H]% 插入两张图片并且并排
		\centering
		\begin{minipage}{0.48\textwidth}
			\centering
			\includegraphics[width=1.1\textwidth]{train_loss_repro_e105.png}
			\caption{\fontsize{10pt}{15pt}\selectfont loss}
		\end{minipage}
		\hspace{0cm}% 图片间距
		\hfill% 撑满整行
		\begin{minipage}{0.48\textwidth}
			\centering
			\includegraphics[width=1.1\textwidth]{repro_e105_easy.pdf}
			\caption{\fontsize{10pt}{15pt}\selectfont easy}
		\end{minipage}
	\end{figure}
	\begin{figure}[H]% 插入两张图片并且并排
		\centering
		\begin{minipage}{0.48\textwidth}
			\centering
			\includegraphics[width=1.1\textwidth]{repro_e105_middle.pdf}
			\caption{\fontsize{10pt}{15pt}\selectfont middle}
		\end{minipage}
		\hspace{0cm}% 图片间距
		\hfill% 撑满整行
		\begin{minipage}{0.48\textwidth}
			\centering
			\includegraphics[width=1.1\textwidth]{repro_e105_hard.pdf}
			\caption{\fontsize{10pt}{15pt}\selectfont hard}
		\end{minipage}
	\end{figure}	
	\subsubsection{PR曲线}
	easy、middle、hard三个验证集的子部分的PR曲线见上

	\section{网络结构改进}
	\subsection{思路来源}
	\begin{figure}[h]% 插入一张图片,H表示浮动环境下的here
		\centering
		\begin{minipage}{1\textwidth}% 小页面尺寸,可自行调节
			\centering
			\includegraphics[width=1.0\textwidth]{Ablation_exp.png}% 图片名称(图片需与tex文件在同一文件夹)
			\caption{\fontsize{10pt}{15pt}\selectfont Ablation experience
		}% 图例
		\end{minipage}
	\end{figure}		
	观察上面作者所做的实验,可以发现在加入 IOU-aware 分支后 AP 提升较大,作者做出的改进主要是在head部分的,boneback和neck部分的提升并不大。
	如果继续改进 head 的话,看了看最近的方向是 ancher-free 的,但是如果将头部换成ancher-free的,似乎就是在复现 FCOS\cite{DBLP:conf/iccv/TianSCH19} 之类的网络。
	所以想保留作者的改进,来对neck部分进行改进,这里使用了 EfficientDet\cite{DBLP:journals/corr/abs-1911-09070} 的思路,将普通的FPN换成BiFPN。\\
	\begin{enumerate}[1.]
		\item 删除那些只有一条输入边的节点。它对旨在融合不同特征的特征网络的贡献很小,删除它网络影响不大,同时简化了双向网络\\
		\item 如果原始输入节点和输出节点处于同一层,在原始输入节点和输出节点之间添加一条额外的边\\
		\item 带权特征融合:学习不同输入特征的重要性,对不同输入特征有区分的融合。设计思路:传统的特征融合往往只是简单的特征图叠加/相加等合并操作,
		而不同的输入特征图具有不同的分辨率,对融合后特征图的贡献也是不同的,
		因此简单的对他们进行相加或叠加处理并不是最佳的操作。可以使用一种简单而高效的加权特融合的机制, $\omega$ 是学习到的参数,用于区分特征融合过程中不同特征的重要程度
		\begin{equation}
			O = \sum_{i} \frac{\omega_i}{\epsilon + \sum_{j} \omega_j} I_i
		\end{equation}
	\end{enumerate}
	
	BiFPN = 新型加强版的PANet(重复双向跨尺度连接) + 带权重的特征融合机制
	\begin{figure}[h]% 插入一张图片,H表示浮动环境下的here
		\centering
		\begin{minipage}{1\textwidth}% 小页面尺寸,可自行调节
			\centering
			\includegraphics[width=1.0\textwidth]{BiFPN.png}% 图片名称(图片需与tex文件在同一文件夹)
			\caption{\fontsize{10pt}{15pt}\selectfont FPN及其变体
		}% 图例
		\end{minipage}
	\end{figure}		

	\subsection{实现细节}
	将 neck 部分原来的 FPN 部分替换为 BiFPN,可能是因为结点数、连接数以及要学习参数的增多,按照原来的batchsize训练开始爆显存,开始时以为是新加的Bifpn.py有bug,后来发现简单的
	减小batchsize就可以解决问题。最后使用batchsize = 2 在两块1080Ti上训练了75轮(同样考虑算力问题)
	\subsection{实验结果分析}
	\begin{center}
		\begin{tabular}{c| c||c|c|c}
			\hline
			模型 & epochs & AP(easy)& AP(middle) & AP(hard)\\
			\hline
			TinaFace(fpn) & 630 & 0.963 & 0.957 &	0.930\\
			\hline
			hgy(fpn) & 105 & 0.940 & 0.931 &	0.899\\
			\hline
			repro(fpn) & 75 & 0.942 & 0.934 &	0.0.891\\
			\hline
			bifpn(bifpn) & 105 & 0.955 & 0.946 &	0.914\\
			\hline
		\end{tabular}
		\end{center}	
	\subsubsection{loss曲线}
	见左下 Figure 8,两者训练过程中loss的波动频率相似,但是neck更换bifpn的网络的波动幅度较大,可以到达更低的loss水平
	\begin{figure}[H]% 插入两张图片并且并排
		\centering
		\begin{minipage}{0.48\textwidth}
			\centering
			\includegraphics[width=1.1\textwidth]{train_loss_e75.png}
			\caption{\fontsize{10pt}{15pt}\selectfont train loss}
		\end{minipage}
		\hspace{0cm}% 图片间距
		\hfill% 撑满整行
		\begin{minipage}{0.48\textwidth}
			\centering
			\includegraphics[width=1.1\textwidth]{bifpn_e75_easy.pdf}
			\caption{\fontsize{10pt}{15pt}\selectfont easy}
		\end{minipage}
	\end{figure}
	\begin{figure}[H]% 插入两张图片并且并排
		\centering
		\begin{minipage}{0.48\textwidth}
			\centering
			\includegraphics[width=1.1\textwidth]{bifpn_e75_middle.pdf}
			\caption{\fontsize{10pt}{15pt}\selectfont middle}
		\end{minipage}
		\hspace{0cm}% 图片间距
		\hfill% 撑满整行
		\begin{minipage}{0.48\textwidth}
			\centering
			\includegraphics[width=1.1\textwidth]{bifpn_e75_hard.pdf}
			\caption{\fontsize{10pt}{15pt}\selectfont hard}
		\end{minipage}
	\end{figure}	
	\subsubsection{PR曲线}
	easy、middle、hard三个验证集的子部分的PR曲线见上。标注TinaFace代表作者630轮neck为FPN,标注hgy代表105轮neck为FPN,
	标注repro代表75轮neck为FPN,标注bifpn代表105轮neck为BiFPN。

	BiFPN为neck的模型在相同轮数(75 epochs)时优于原模型;有趣的是原模型在75轮时在easy、middle部分优于105轮时的情况,105轮时应该是为了学习hard部分,
	放弃了一部分的easy和middle的AP得分。
	\section{实验总结}
	\begin{enumerate}[1.]
		\item 要多关注一些其它的研究方向,吸收前人的成果,站在巨人的肩膀上才能看的更远。
		\item 大模型、大数据集是机器学习的潮流,找要复现的paper时发现几乎全部是大模型大数据集的,需要一定的资源才能继续深度学习方面的学习
		\item 感觉深度学习已经变成了一门实验学科,怪玄学的,有些paper看的挺不明不白,参考网上的博客感觉也还是雾里看花,缺乏系统性的学习
	\end{enumerate}
	\section[工程结构]{工程结构\footnote{https://github.com/guangyid/dlp\_hw3}}
	\begin{enumerate}
		\item 	report文件夹是与工程无关的,其中的log文件是训练时生成的loss文件,graph.ipynb是根据这些log文件生成可视化loss曲线的,jpg2lmdb.ipynb是转化数据集为lmdb格式时使用的。
		\item 	eval\_tools文件夹是WIDERFace数据集评估PR曲线的matlab程序
		\item 	其余文件夹均与训练相关
		\item 	我更改的大段代码会单独一行注释上 --hgy--
	\end{enumerate}

\bibliographystyle{plain}
\bibliography{tinaface,FCOS,EfficientDet}
		
\end{document}% 结束文档编辑,后面写啥都编译不出来
% 	\subsection{摆的理论基础}
% 	   \subsubsection{Evildoer的摆理论}
%  	  大本钟下寄快递,上面开摆下面寄
%   	 \subsubsection{摆理论的完善与发展}
% 	\subsection{摆的实际应用}
% 		\begin{enumerate}[1.]% 列举时编号
%   		 	\item 啊对
% 		    \begin{enumerate}[(a)]% 次级序号
% 			\item 太对辣
% 			\item 好对捏
%    		\end{enumerate}
%    		\item 啊对对
% 	   	 \item 啊对对对\footnote{变成光守护麻衣学姐}% 脚注
%    	\end{enumerate}

% % 	\section{摆烂三阶段}
% % 	至臻无双
% % 	\begin{figure}[H]% 插入一张图片,H表示浮动环境下的here
% % 	\centering
% % 	\begin{minipage}{0.83\textwidth}% 小页面尺寸,可自行调节
% % 		\centering
% % 		\includegraphics[width=1.0]% 图片尺寸,可自行调节
% % 		% \textwidth]{lqx}% 图片名称(图片需与tex文件在同一文件夹)
% % 		% \caption{\fontsize{10pt}{15pt}\selectfont 单图}% 图例
% % 	\end{minipage}
% % \end{figure}

% \begin{figure}[H]% 插入两张图片并且并排
% 	\centering
% 	\begin{minipage}{0.48\textwidth}
% 		\centering
% 		% \includegraphics[width=0.83\textwidth]{title部分效果}
% 		\caption{\fontsize{10pt}{15pt}\selectfont 俩图}
% 	\end{minipage}
% 	\hspace{0cm}% 图片间距
% 	\hfill% 撑满整行
% 	\begin{minipage}{0.48\textwidth}
% 		\centering
% 		% \includegraphics[width=0.83\textwidth]{框架}
% 		\caption{\fontsize{10pt}{15pt}\selectfont 俩图}
% 	\end{minipage}
% \end{figure}
% \begin{equation}% 单个公式
% 	C_0=\frac{2V_1\text{arcth}\left[ \frac{\left( L+R_1-R_2 \right) \left( L-R_1-R_2 \right)}{\left( L+R_1+R_2 \right) \left( L-R_1+R_2 \right)} \right] ^{\frac{1}{2}}}{\text{arch}\left( \frac{L^2-R_{1}^{2}-R_{2}^{2}}{2R_1R_2} \right)}+\frac{2V_2\text{arcth}\left[ \frac{\left( L+R_2-R_1 \right) \left( L-R_1-R_2 \right)}{\left( L+R_1+R_2 \right) \left( L-R_2+R_1 \right)} \right] ^{\frac{1}{2}}}{\text{arch}\left( \frac{L^2-R_{1}^{2}-R_{2}^{2}}{2R_1R_2} \right)}
% \end{equation}

% \begin{equation}
% 	\centering
% 	\begin{split}% 多个公式
% 		A_0&=\frac{V_2-V_1}{\ln \dfrac{R_{2}^{'}}{R_{1}^{'}}}\\
% 		C_0&=\frac{V_1\ln R_{2}^{'}-V_2\ln R_{1}^{'}}{\ln \dfrac{R_{2}^{'}}{R_{1}^{'}}}
% 	\end{split}
% \end{equation}

% \begin{align}% 对齐公式
% 		A_0&=3c+6666\\% 注意换行
% 		&=369
% \end{align}
